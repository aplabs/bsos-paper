\documentclass[english]{article}
\usepackage[T1]{fontenc}
\usepackage[latin9]{inputenc}
\usepackage{listings}
\usepackage{geometry}
\geometry{verbose,tmargin=1.47cm,bmargin=1.47cm,lmargin=1.47cm,rmargin=1.47cm}

\makeatletter
\pagestyle{empty}
\usepackage{tikz}
\usetikzlibrary{arrows,shadows}
\usepackage[underline=true,rounded corners=false]{pgf-umlsd}
\usepackage{graphicx}

\makeatother

\usepackage{babel}
\begin{document}

\tikzstyle{decision} = [diamond, draw, fill=blue!20, 
    text width=4.5em, text badly centered, node distance=3cm, inner sep=0pt]
\tikzstyle{block} = [rectangle, draw, fill=blue!20, 
    text width=5em, text centered, rounded corners, minimum height=4em]
\tikzstyle{line} = [draw, -latex']
\tikzstyle{cloud} = [draw, ellipse,fill=red!20, node distance=3cm,
    minimum height=2em]

\title{BSOS}
\author{Alexander Gould, William Woodruff, Pawe\l{} Czarnecki, Reyna Shaskan}
\date{Post-AP Robotics Final Project}
\maketitle

\begin{abstract}

\end{abstract}

\section{Booting Up}
BSOS's bootstrapping process can be classified by architecture. On x86 machines, specifically i386 and later, BSOS begins in 16-bit real mode. The strapping sector, which is 512 bytes long, then loads a general descriptor table, or GDT, into memory. That descriptor table is then used to switch the CPU into 32-bit protected mode, whereupon the kernel is loaded.

\begin{figure}[h!]

\caption{Figure 1: BSOS's x86 booting procedure and sector layout.}
\end{figure}

\section{Memory Management}

\begin{lstlisting}
Pawe's stuff goes here.
\end{lstlisting}



\section{Running the OS}

\begin{lstlisting}
Reyna's stuff goes here.
\end{lstlisting}



\section{Interfacing With Hardware}

\begin{lstlisting}
Gould's stuff goes here.
\end{lstlisting}

\end{document}
