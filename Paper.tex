\documentclass[english]{article}
\usepackage[T1]{fontenc}
\usepackage[latin9]{inputenc}
\usepackage{listings}
\usepackage{geometry}
\geometry{verbose,tmargin=1.47cm,bmargin=1.47cm,lmargin=1.47cm,rmargin=1.47cm}

\makeatletter
\pagestyle{empty}
\usepackage{tikz}
\usetikzlibrary{arrows,shadows} % for pgf-umlsd
\usepackage[underline=true,rounded corners=false]{pgf-umlsd}

\makeatother

\usepackage{babel}
\begin{document}

\title{BSOS}


\author{Alexander Gould, William Woodruff, Pawe\l{} Czarnecki, Reyna Shaskan}


\date{Post-AP Robotics Final Project}
\maketitle
\begin{abstract}
For our final project, we decided to play around with low level programming
and system design, so we decided to make our own operating system
for the Cortex! It's inspired by {*}NIX systems, and runs on x86 processors
(like the ones on our computers) and ARM processors (like the ones
in the Cortex). It has support for traditional keyboard input and
screen output and is also able to interface with the motors and sensors
on the VeX system. It bosts far less bloat and faster response time
than the pre-loaded OS on the Cortex. It's called BSOS because the
letters sound nice and not for any other reason whatsoever.
\end{abstract}

\part{Booting Up}

\begin{lstlisting}
Will's stuff goes here.
\end{lstlisting}



\part{Memory Management}

\begin{lstlisting}
Pawe's stuff goes here.
\end{lstlisting}



\part{Running the OS}

\begin{lstlisting}
Reyna's stuff goes here.
\end{lstlisting}



\part{Interfacing With Hardware}
There are 2 main components to input and output for this operating system, keyboard and screen, and motors and sensors.
\paragraph{Keyboard and Screen}
Information is se4nt from the keyboard to the processor via a standard protocol. Inside any commercial keyboard is a tiny computer chip
called a {\it microcontroller}. When a key is pressed down, the microcontroller receives the signal and starts a short countdown timer. If
your finger is still pressing down on the key after more than a millisecond, the microcontroller looks up a numerical value for the key
and saves it in its small memory bank.
\\The data is then sent to the computer along a cable, usually USB. At any given moment, power
is passing through this cable to supply power to the keyboard, as well as to synchronize the keyboards internal timer. When the
computer receives this data is sent to a program known as the {\it driver} for processing.
\paragraph{Sensors and Motors}

\end{document}
