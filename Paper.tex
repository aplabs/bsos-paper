\documentclass[english]{article}
\usepackage[T1]{fontenc}
\usepackage[latin9]{inputenc}
\usepackage{listings}
\usepackage{geometry}
\geometry{verbose,tmargin=1.47cm,bmargin=1.47cm,lmargin=1.47cm,rmargin=1.47cm}

\makeatletter
\pagestyle{empty}
\usepackage{tikz}
\usetikzlibrary{arrows,shadows} % for pgf-umlsd
\usepackage[underline=true,rounded corners=false]{pgf-umlsd}

\makeatother

\usepackage{babel}
\begin{document}

\title{BSOS}


\author{Alexander Gould, William Woodruff, Pawe\l{} Czarnecki, Reyna Shaskan}


\date{Post-AP Robotics Final Project}
\maketitle
\begin{abstract}
For our final project, we decided to play around with low level programming
and system design, so we decided to make our own operating system
for the Cortex! It's inspired by {*}NIX systems, and runs on x86 processors
(like the ones on our computers) and ARM processors (like the ones
in the Cortex). It has support for traditional keyboard input and
screen output and is also able to interface with the motors and sensors
on the VeX system. It bosts far less bloat and faster response time
than the pre-loaded OS on the Cortex. It's called BSOS because the
letters sound nice and not for any other reason whatsoever.
\end{abstract}

\part{Booting Up}

\begin{lstlisting}
Will's stuff goes here.
\end{lstlisting}



\part{Memory Management}

\begin{lstlisting}
Pawe's stuff goes here.
\end{lstlisting}



\part{Running the OS}

\begin{lstlisting}
Reyna's stuff goes here.
\end{lstlisting}



\part{Interfacing With Hardware}

\begin{lstlisting}
Gould's stuff goes here.
\end{lstlisting}

\end{document}
